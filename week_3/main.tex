% -------------------------------------
%               Setup
% -------------------------------------
\documentclass[t, 24pt]{beamer}
\usepackage{lipsum}
\usepackage{listings} % for code
\usepackage{etoolbox} % For \ifstrempty00
\newcommand{\weekcount}{3} % \def\weekcount{3}

% Setup > KU Setup
\usepackage{KUstyle}
\toplinje{Meeting | Week \weekcount} % The text at top. Remove the command if no text is desired

% Code
\usepackage{fancyvrb}
\VerbatimEnvironment

% -------------------------------------
%               Code
% -------------------------------------
\definecolor{bgcolor}{RGB}{248,248,248} % Light gray background
\definecolor{keywordcolor}{RGB}{0,0,160} % Dark blue keywords
\definecolor{stringcolor}{RGB}{163,21,21} % Dark red strings
\definecolor{commentcolor}{RGB}{0,128,0} % Dark green comments
\definecolor{numbercolor}{RGB}{128,0,128} % Purple line numbers
\definecolor{promptcolor}{RGB}{64,64,64} % Dark gray prompts

% Custom style for Python code
\lstdefinestyle{professionalPython}{
    language=Python,
    backgroundcolor=\color{bgcolor},
    basicstyle=\linespread{1.1}\tiny\ttfamily,
    keywordstyle=\color{keywordcolor}\bfseries,
    commentstyle=\color{commentcolor}\itshape,
    stringstyle=\color{stringcolor},
    numberstyle=\tiny\color{numbercolor},
    identifierstyle=\color{black},
    tabsize=4,
    showstringspaces=false,
    numbers=left,
    stepnumber=1,
    numbersep=10pt,
    frame=single,
    framerule=0.5pt,
    rulecolor=\color{gray},
    breaklines=true,
    captionpos=b,
    xleftmargin=15pt,
    xrightmargin=5pt,
    aboveskip=10pt,
    belowskip=-10pt,
    morekeywords={def, return, class, from, import, as, if, elif, else, while, for, in, try, except, finally, with, yield, lambda, global, nonlocal, assert, pass, break, continue, raise},
    literate=*{:}{{\textcolor{keywordcolor}{:}}}1,
}


% -------------------------------------
% Command with question slides counter
% -------------------------------------

\newcounter{subjectNumber}
\setcounter{subjectNumber}{0}

\makeatletter
\newcommand{\resetsubject}{%
\stepcounter{subjectNumber}%
\@ifundefined{c@currentQSlide\thesubjectNumber}{%
\newcounter{currentQSlide\thesubjectNumber}%
}{}%
\setcounter{currentQSlide\thesubjectNumber}{0}%
}

\newcommand{\incrementSlideCnt}{%
\stepcounter{currentQSlide\thesubjectNumber}%
}

\AtEndDocument{%
\@tempcnta=1 \loop\ifnum\@tempcnta<\numexpr\value{subjectNumber}+1 \immediate\write\@auxout{%
\string\@ifundefined{c@totalQSlide\the\@tempcnta}{%
\string\newcounter{totalQSlide\the\@tempcnta}%
}{}%
\string\setcounter{totalQSlide\the\@tempcnta}{\the\value{currentQSlide\the\@tempcnta}}%
}%
\advance\@tempcnta by1 \repeat }
\makeatother



\newenvironment{numberedSlide}[2][]{%
  \ifstrempty{#1}{%
    % If no optional argument => just [hoved]
    \begin{frame}[hoved]
  }{%
    % If there *is* an optional argument => [containsverbatim,hoved]
    \begin{frame}[#1,hoved]
  }%
  \incrementSlideCnt
  \frametitle{#2 (\arabic{currentQSlide\thesubjectNumber}/\arabic{totalQSlide\thesubjectNumber})}
}{%
  \end{frame}
}


%\newenvironment{numberedSlide}[1]{%
%  \begin{frame}[hoved]
%    \incrementSlideCnt
%    \frametitle{#1 (\arabic{currentQSlide\thesubjectNumber}/\arabic{totalQSlide\thesubjectNumber})}
%}{%
%  \end{frame}
%}


% -------------------------------------
%                Setup
% -------------------------------------
\resetsubject % Initialize the first subject

% -------------------------------------
%                TiKz
% -------------------------------------

\usepackage{tikz}
\usetikzlibrary{shapes.geometric, arrows}

\tikzstyle{startstop}
=
[rectangle,
rounded
corners,
minimum
width=3.5cm,
minimum
height=1cm,
text
centered,
draw=black,
fill=blue!20]
\tikzstyle{process}
=
[rectangle,
minimum
width=3.5cm,
minimum
height=1cm,
text
centered,
draw=black,
fill=green!20]
\tikzstyle{arrow}
=
[thick,->,>=stealth]

\begin{document}

    % Intro slides.
    { \setbeamertemplate{background}{\includegraphics[width=\paperwidth,height=\paperheight]{KU/forside.pdf}} \begin{frame}\begin{textblock*}{\textwidth}(0\textwidth,0.1\textheight) \begin{beamercolorbox}[wd=6.3cm,ht=7.7cm,sep=0.5cm]{hvidbox} \fontsize{4}{10}\fontfamily{ptm}\selectfont \textls[200]{UNIVERSITY OF COPENHAGEN} \noindent\textcolor{KUrod}{\rule{5.3cm}{0.4pt}}\end{beamercolorbox}\end{textblock*} \begin{textblock*}{\textwidth}(0\textwidth,0.1\textheight) \begin{beamercolorbox}[wd=6.3cm,sep=0.5cm]{hvidbox} \Large \textcolor{KUrod}{Meeting - Week \weekcount} \vspace{0.5cm} \par \large Progress, Challenges \& Next steps \vspace{0.5cm} \par \textcolor{gray}{\scriptsize Simon Winther \& Hjalte Bjoernstrup, Department of Computer Science, University of Copenhagen (DIKU), \today.}\end{beamercolorbox}\end{textblock*} \begin{textblock}{1}(6,11.44) \includegraphics[width=1cm]{KU/KU-logo.png}\end{textblock}\end{frame} }

    % ----------------------------------------------------
    %                       Slide 1
    % ----------------------------------------------------
    \begin{numberedSlide}{Questions?} 
        Should we train a separate model for each organ (e.g., liver, brain, prostate) to maximize specialization, or would a single model trained on all datasets generalize well across different organs? Alternatively, would pretraining on multiple organs and fine-tuning per organ be a better approach?
        \\~\\
        The LiTS preprocessing script resizes images to 320×320 and creates three-channel
        inputs using neighboring slices.
        \begin{itemize}
            \item Should we crop/resize our dataset to match this format, or can
                we keep the original resolution?

            \item How should we handle three-channel input if the number of slices
                is not a multiple of 3?

            \item Are there any strict formatting requirements we need to follow
                for U-Net 3+?
        \end{itemize}
    \end{numberedSlide}

    % ----------------------------------------------------
    %                       Slide 2
    % ----------------------------------------------------
    \begin{numberedSlide}{Questions?} 
        \textbf{Handling 4D NIfTI Files in U-Net 3+}

        Some .nii files contain an extra dimension, making them 4D (e.g.,
        512×512×49×4).

        \begin{itemize}
            \item How should we handle the extra dimension? Should we select a
                specific channel, merge them, or treat them as separate inputs?

            \item If selecting a single channel, which one is most relevant for
                segmentation?

            \item Should we preprocess 4D files differently from standard 3D volumes?
        \end{itemize}

        \textbf{How do we fairly compare this U-net 3+, with our multi-scaling
        approach if we need to crop images differently/more in U-net 3+, than
        our multi-scaling model approach, since then the different cropping seem
        to hold a big factor for the final score.}
    \end{numberedSlide}

    % ----------------------------------------------------
    %                       Slide 3
    % ----------------------------------------------------
    \begin{numberedSlide}{Questions?}
        
    \end{numberedSlide}

\end{document}
