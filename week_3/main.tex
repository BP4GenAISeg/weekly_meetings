% -------------------------------------
%               Setup
% -------------------------------------
\documentclass[t, 24pt]{beamer}
\usepackage{lipsum}
\usepackage{listings} % for code
\usepackage{etoolbox} % For \ifstrempty00
\newcommand{\weekcount}{3} % \def\weekcount{3}

% Setup > KU Setup
\usepackage{KUstyle}
\toplinje{Meeting | Week \weekcount} % The text at top. Remove the command if no text is desired

% Code
\usepackage{fancyvrb}
\VerbatimEnvironment

% -------------------------------------
%               Code
% -------------------------------------
\definecolor{bgcolor}{RGB}{248,248,248} % Light gray background
\definecolor{keywordcolor}{RGB}{0,0,160} % Dark blue keywords
\definecolor{stringcolor}{RGB}{163,21,21} % Dark red strings
\definecolor{commentcolor}{RGB}{0,128,0} % Dark green comments
\definecolor{numbercolor}{RGB}{128,0,128} % Purple line numbers
\definecolor{promptcolor}{RGB}{64,64,64} % Dark gray prompts

% Custom style for Python code
\lstdefinestyle{professionalPython}{ language=Python, backgroundcolor=
\color{bgcolor}
, basicstyle=\linespread{1.1}\tiny\ttfamily, keywordstyle=
\color{keywordcolor}
\bfseries, commentstyle=
\color{commentcolor}
\itshape, stringstyle=
\color{stringcolor}
, numberstyle=\tiny
\color{numbercolor}
, identifierstyle=
\color{black}
, tabsize=4, showstringspaces=false, numbers=left, stepnumber=1, numbersep=10pt,
frame=single, framerule=0.5pt, rulecolor=
\color{gray}
, breaklines=true, captionpos=b, xleftmargin=15pt, xrightmargin=5pt, aboveskip=10pt,
belowskip=-10pt, morekeywords={def, return, class, from, import, as, if, elif, else, while, for, in, try, except, finally, with, yield, lambda, global, nonlocal, assert, pass, break, continue, raise},
literate=*{:}{{\textcolor{keywordcolor}{:}}}1, }

% -------------------------------------
% Command with question slides counter
% -------------------------------------

\newcounter{subjectNumber}
\setcounter{subjectNumber}{0}

\makeatletter
\newcommand{\resetsubject}{%
\stepcounter{subjectNumber}%
\@ifundefined{c@currentQSlide\thesubjectNumber}{%
\newcounter{currentQSlide\thesubjectNumber}%
}{}%
\setcounter{currentQSlide\thesubjectNumber}{0}%
}

\newcommand{\incrementSlideCnt}{%
\stepcounter{currentQSlide\thesubjectNumber}%
}

\AtEndDocument{%
\@tempcnta=1 \loop\ifnum\@tempcnta<\numexpr\value{subjectNumber}+1 \immediate\write\@auxout{%
\string\@ifundefined{c@totalQSlide\the\@tempcnta}{%
\string\newcounter{totalQSlide\the\@tempcnta}%
}{}%
\string\setcounter{totalQSlide\the\@tempcnta}{\the\value{currentQSlide\the\@tempcnta}}%
}%
\advance\@tempcnta by1 \repeat }
\makeatother

\newenvironment{numberedSlide}[2][]{%
% Choose frame option
\ifstrempty{#1}{%
\begin{frame}[hoved] }{%
\begin{frame}[#1,hoved] }%
\incrementSlideCnt
% Fallback check for totalQSlide
\frametitle{%
#2 \space \ifcsname c@totalQSlide\thesubjectNumber\endcsname (\arabic{currentQSlide\thesubjectNumber}/\arabic{totalQSlide\thesubjectNumber})
\else (\arabic{currentQSlide\thesubjectNumber}/?) \fi }
}{%
\end{frame} }

%\newenvironment{numberedSlide}[1]{%
%  \begin{frame}[hoved]
%    \incrementSlideCnt
%    \frametitle{#1 (\arabic{currentQSlide\thesubjectNumber}/\arabic{totalQSlide\thesubjectNumber})}
%}{%
%  \end{frame}
%}

% -------------------------------------
%                Setup
% -------------------------------------
\resetsubject % Initialize the first subject

% -------------------------------------
%                TiKz
% -------------------------------------

\usepackage{tikz}
\usetikzlibrary{shapes.geometric, arrows}

\tikzstyle{startstop}
=
[rectangle,
rounded
corners,
minimum
width=3.5cm,
minimum
height=1cm,
text
centered,
draw=black,
fill=blue!20]
\tikzstyle{process}
=
[rectangle,
minimum
width=3.5cm,
minimum
height=1cm,
text
centered,
draw=black,
fill=green!20]
\tikzstyle{arrow}
=
[thick,->,>=stealth]

\begin{document}
  % Intro slides.
  { \setbeamertemplate{background}{\includegraphics[width=\paperwidth,height=\paperheight]{KU/forside.pdf}} \begin{frame}\begin{textblock*}{\textwidth}(0\textwidth,0.1\textheight) \begin{beamercolorbox}[wd=6.3cm,ht=7.7cm,sep=0.5cm]{hvidbox} \fontsize{4}{10}\fontfamily{ptm}\selectfont \textls[200]{UNIVERSITY OF COPENHAGEN} \noindent\textcolor{KUrod}{\rule{5.3cm}{0.4pt}}\end{beamercolorbox}\end{textblock*} \begin{textblock*}{\textwidth}(0\textwidth,0.1\textheight) \begin{beamercolorbox}[wd=6.3cm,sep=0.5cm]{hvidbox} \Large \textcolor{KUrod}{Meeting - Week \weekcount} \vspace{0.5cm} \par \large Progress, Challenges \& Next steps \vspace{0.5cm} \par \textcolor{gray}{\scriptsize Simon Winther \& Hjalte Bjoernstrup, Department of Computer Science, University of Copenhagen (DIKU), \today.}\end{beamercolorbox}\end{textblock*} \begin{textblock}{1}(6,11.44) \includegraphics[width=1cm]{KU/KU-logo.png}\end{textblock}\end{frame} }

  % ----------------------------------------------------
  %                       Slide 1
  % ----------------------------------------------------
  \begin{numberedSlide}
    [allowframebreaks]{Questions?}
    \begin{enumerate}
      \item When processing a 3D medical image stored as a \texttt{.NII} file, we
        obtain a set of 49 two-dimensional images of resolution $512 \times 512$.
        How does this representation relate to a true 3D representation? We think
        this is what is causing some confusion for us—what exactly do these 49 slices
        represent? They look very similar, and it does not seem like they capture
        different perspectives of the brain or tumor. Rather, it seems like the medical
        image is just shifting slightly, which makes it difficult to fully
        understand where the 3D aspect comes into play. Because the way that we've
        understood it was that, for example, the MRI slices could be moving from
        top to bottom, with each slice capturing a specific layer of the brain
        at that depth. In the end, stacking all slices together would form a
        complete volumetric 3D representation of the brain, and so why would the
        input channel not be the whole \textbf{49 channels}?

      \item Given that each \texttt{.NII} file produces a sequence of 2D slices,
        what exactly does each slice represent in a 3D context? Does the stack
        of 2D slices reconstruct a volumetric image, or is additional processing
        required? We know that you mentioned that maybe we should move to a
        standard 3D U-Net, but we are a bit confused as to why U-Net 3+ is not
        considered 3D. If the input has three slices (or three channels), what
        makes it strictly 2D? This distinction is a key part of our confusion.

      \item If U-Net 3+ takes only three input channels per pass, how does this limit
        its ability to fully utilize the 3D information present in the dataset?
        Would a fully 3D input require taking all 49 slices at once, forming a $5
        12 \times 512 \times 49$ input tensor? (\textbf{this overlaps with
        Question 1}). The reason why we are asking this is that you suggested we
        should go for a standard 3D U-Net, with multiscale (multiresolution),
        deep supervision and plain skip connections leaving out the fullscale
        skip connections included in the unet3p framework. In this case, if we
        took the entire $512 \times 512 \times 49$ as input (i.e., all 49 slices
        as channels), would that then be considered fully 3D, or is there
        another aspect that makes a model fundamentally 3D? This connects with our
        confusion in questions one and two.

      \item There was some confusion in the group regarding the definitions of certain
        terms. The full-scale skip connections refer to the concatenation of
        encoder outputs to lower-resolution decoder layers, whereas in a standard
        U-Net, skip connections link encoder and decoder layers at the same
        depth (e.g., depth 0 in the encoder connects to depth 0 in the decoder,
        depth 1 to depth 1, etc.). \textbf{Is this correctly understood?}

        \\~\\

        The multi-scaling mechanism, however, is entirely separate from the full-scale
        skip connection. We are removing the full-scale skip connections
        completely because they introduce additional complexity. Instead, the
        multi-scaling approach we want to implement is based on multi-resolution
        learning. The idea is that at different depths, the network should be
        able to predict what the appropriate input channels should look like. For
        example, at a depth of 256, then at a depth of 128, the input should be
        adjusted accordingly. Since each convolutional layer with pooling operations
        changes the number of channels based on the kernel sizes and pooling factors,
        we aim to train a neural network to predict the lower-resolution
        features that would best match the expected structure at each depth. This
        way, we can retain meaningful features across different resolutions
        without relying on full-scale skip connections. \textbf{Is this also a
        correct understanding?}

      \item If you still believe it would be best for us to switch to a completely
        different 3D U-Net model, do you have a specific GitHub repository or
        paper in mind that you could share? We would like to review it first to better
        understand the differences. Additionally, would such a model take all slices
        at once as input (i.e., $512 \times 512 \times 49$), or how should we
        process the dataset is where the confusion lies as mentioned above.
    \end{enumerate}
  \end{numberedSlide}
\end{document}