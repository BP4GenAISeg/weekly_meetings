% -------------------------------------
%               Setup
% -------------------------------------
\documentclass[t, 24pt]{beamer}
\usepackage{lipsum}
\usepackage{listings} % for code
\usepackage{etoolbox} % For \ifstrempty00
\newcommand{\weekcount}{3} % \def\weekcount{3}

% Setup > KU Setup
\usepackage{KUstyle}
\toplinje{Meeting | Week \weekcount} % The text at top. Remove the command if no text is desired

% Code
\usepackage{fancyvrb}
\VerbatimEnvironment

% -------------------------------------
%               Code
% -------------------------------------
\definecolor{bgcolor}{RGB}{248,248,248} % Light gray background
\definecolor{keywordcolor}{RGB}{0,0,160} % Dark blue keywords
\definecolor{stringcolor}{RGB}{163,21,21} % Dark red strings
\definecolor{commentcolor}{RGB}{0,128,0} % Dark green comments
\definecolor{numbercolor}{RGB}{128,0,128} % Purple line numbers
\definecolor{promptcolor}{RGB}{64,64,64} % Dark gray prompts

% Custom style for Python code
\lstdefinestyle{professionalPython}{
    language=Python,
    backgroundcolor=\color{bgcolor},
    basicstyle=\linespread{1.1}\tiny\ttfamily,
    keywordstyle=\color{keywordcolor}\bfseries,
    commentstyle=\color{commentcolor}\itshape,
    stringstyle=\color{stringcolor},
    numberstyle=\tiny\color{numbercolor},
    identifierstyle=\color{black},
    tabsize=4,
    showstringspaces=false,
    numbers=left,
    stepnumber=1,
    numbersep=10pt,
    frame=single,
    framerule=0.5pt,
    rulecolor=\color{gray},
    breaklines=true,
    captionpos=b,
    xleftmargin=15pt,
    xrightmargin=5pt,
    aboveskip=10pt,
    belowskip=-10pt,
    morekeywords={def, return, class, from, import, as, if, elif, else, while, for, in, try, except, finally, with, yield, lambda, global, nonlocal, assert, pass, break, continue, raise},
    literate=*{:}{{\textcolor{keywordcolor}{:}}}1,
}


% -------------------------------------
% Command with question slides counter
% -------------------------------------

\newcounter{subjectNumber}
\setcounter{subjectNumber}{0}

\makeatletter
\newcommand{\resetsubject}{%
\stepcounter{subjectNumber}%
\@ifundefined{c@currentQSlide\thesubjectNumber}{%
\newcounter{currentQSlide\thesubjectNumber}%
}{}%
\setcounter{currentQSlide\thesubjectNumber}{0}%
}

\newcommand{\incrementSlideCnt}{%
\stepcounter{currentQSlide\thesubjectNumber}%
}

\AtEndDocument{%
\@tempcnta=1 \loop\ifnum\@tempcnta<\numexpr\value{subjectNumber}+1 \immediate\write\@auxout{%
\string\@ifundefined{c@totalQSlide\the\@tempcnta}{%
\string\newcounter{totalQSlide\the\@tempcnta}%
}{}%
\string\setcounter{totalQSlide\the\@tempcnta}{\the\value{currentQSlide\the\@tempcnta}}%
}%
\advance\@tempcnta by1 \repeat }
\makeatother



\newenvironment{numberedSlide}[2][]{%
  % Choose frame option
  \ifstrempty{#1}{%
    \begin{frame}[hoved]
  }{%
    \begin{frame}[#1,hoved]
  }%
  \incrementSlideCnt
  % Fallback check for totalQSlide
  \frametitle{%
    #2 \space
    \ifcsname c@totalQSlide\thesubjectNumber\endcsname (\arabic{currentQSlide\thesubjectNumber}/\arabic{totalQSlide\thesubjectNumber})
    \else
      (\arabic{currentQSlide\thesubjectNumber}/?)
    \fi
  }
}{%
  \end{frame}
}



%\newenvironment{numberedSlide}[1]{%
%  \begin{frame}[hoved]
%    \incrementSlideCnt
%    \frametitle{#1 (\arabic{currentQSlide\thesubjectNumber}/\arabic{totalQSlide\thesubjectNumber})}
%}{%
%  \end{frame}
%}


% -------------------------------------
%                Setup
% -------------------------------------
\resetsubject % Initialize the first subject

% -------------------------------------
%                TiKz
% -------------------------------------

\usepackage{tikz}
\usetikzlibrary{shapes.geometric, arrows}

\tikzstyle{startstop}
=
[rectangle,
rounded
corners,
minimum
width=3.5cm,
minimum
height=1cm,
text
centered,
draw=black,
fill=blue!20]
\tikzstyle{process}
=
[rectangle,
minimum
width=3.5cm,
minimum
height=1cm,
text
centered,
draw=black,
fill=green!20]
\tikzstyle{arrow}
=
[thick,->,>=stealth]

\begin{document}
    % Intro slides.
    { \setbeamertemplate{background}{\includegraphics[width=\paperwidth,height=\paperheight]{KU/forside.pdf}} \begin{frame}\begin{textblock*}{\textwidth}(0\textwidth,0.1\textheight) \begin{beamercolorbox}[wd=6.3cm,ht=7.7cm,sep=0.5cm]{hvidbox} \fontsize{4}{10}\fontfamily{ptm}\selectfont \textls[200]{UNIVERSITY OF COPENHAGEN} \noindent\textcolor{KUrod}{\rule{5.3cm}{0.4pt}}\end{beamercolorbox}\end{textblock*} \begin{textblock*}{\textwidth}(0\textwidth,0.1\textheight) \begin{beamercolorbox}[wd=6.3cm,sep=0.5cm]{hvidbox} \Large \textcolor{KUrod}{Meeting - Week \weekcount} \vspace{0.5cm} \par \large Progress, Challenges \& Next steps \vspace{0.5cm} \par \textcolor{gray}{\scriptsize Simon Winther \& Hjalte Bjoernstrup, Department of Computer Science, University of Copenhagen (DIKU), \today.}\end{beamercolorbox}\end{textblock*} \begin{textblock}{1}(6,11.44) \includegraphics[width=1cm]{KU/KU-logo.png}\end{textblock}\end{frame} }

    % ----------------------------------------------------
    %                       Slide 1
    % ----------------------------------------------------
    \begin{numberedSlide}[containsverbatim]{Questions?} 
            
             \begin{itemize}
                \item can we just change 3 => 49 and in theory it could work for 3D? 
                    \begin{lstlisting}[style=professionalPython]
""" Encoder """
self.e1 = encoder_block(3, 64)
self.e2 = encoder_block(64, 128)
self.e3 = encoder_block(128, 256)
self.e4 = encoder_block(256, 512)
            \end{lstlisting}
            \vspace{1em}

                \item We also have a lot of dependency problem, since it from at least 4 years ago, where most of the code was build and was wondering if any realistic way to find their dependencies versions? or if it wouldn't just be easier to just create it ourself from scratch?

                
                
             \end{itemize}
            
            
            
    \end{numberedSlide}


       \begin{numberedSlide}[containsverbatim]{Questions?} 
            
             \begin{itemize}
                \item This is the whole network, how do we create a whole new network from this, it seem very inflexible. (multiscaling)
                    \begin{lstlisting}[style=professionalPython]
net = DynUNet(
        spatial_dims=3,
        in_channels=in_channels,
        out_channels=n_class,
        kernel_size=kernels,
        strides=strides,
        upsample_kernel_size=strides[1:],
        norm_name="instance",
        deep_supervision=True,
        deep_supr_num=deep_supr_num[task_id],
    )
            \end{lstlisting}
            \vspace{1em}

               

                
                
             \end{itemize}
            
            
            
    \end{numberedSlide}


\end{document}
